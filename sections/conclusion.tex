\section{Conclusion}
\justifying

This project demonstrates that boot-time optimization is feasible, measurable, and repeatable when it is treated as an engineering workflow rather than random tweaking. The optimization scope was intentionally limited to the two phases that are directly controllable and consistently measurable across the full test matrix: the \textbf{Loader} and \textbf{Kernel} phases reported by \texttt{systemd-analyze}. Each change was evaluated using five cold-boot trials before and after modification, and the results were summarized with mean and sample standard deviation to make variability visible.\\

On native x86 systems, migrating from \textbf{GRUB to systemd-boot} reliably reduced Loader time by simplifying the pre-kernel path and using minimal, explicit boot entries. Kernel configuration work followed a dependency-aware trimming strategy with USB safety constraints, where aggressive removal that is acceptable on NVMe can break early root discovery on removable media. For this reason, some changes were applied only to NVMe setups, while USB setups retained a broader early-driver surface for stability. When initramfs-related decisions are discussed, they are included only because they directly influence early driver availability and therefore the practical Kernel path during boot (especially on USB).The Debian USB optimized set demonstrates that meaningful Loader and Kernel improvements can be achieved under removable-media constraints, and total boot time is heavily dependent to USB storage I/O behavior.\\

In the Parallels ARM virtual machine, the highest-yield improvement came from identifying and removing a single dominant kernel stall source rather than relying on general trimming alone. The \texttt{libahci.ignore\_sss=1} mitigation addressed a hypervisor-specific AHCI probing behavior that serialized unused port scans, producing the largest Kernel time reduction across VM configurations. Kernel trimming provided a smaller but consistent secondary gain. This confirms a key lesson of the VM track: profiling-guided fixes aimed at the real bottleneck can outperform broad minimization. Parallels VM, being one of the most popular and advanced virtualization softwares on MacOS, could easily reduce Linux boot times by approximately 1.5 seconds on all Linux guests. As a more conservative and safer way to implement this change, Parallels can query if connected SATA devices are external, to automatically decide when to request staggered spin-up.\\

Overall, the results support three practical takeaways. First, bootloader simplification is a low-risk and repeatable way to reduce Loader time. Second, kernel trimming is effective when guided by dependency awareness and validated through cold-boot measurements, but USB boots require conservative decisions to preserve early root discovery. Third, virtualization environments can introduce platform-specific kernel bottlenecks that must be addressed with targeted parameters rather than generic configuration reduction. If future work is considered and total boot time becomes the target, should focus on controlling userspace variability and explaining (or eliminating) the Debian NVMe regression factors while keeping the same measurement discipline.
