\section{Results}
\justifying

\subsection{Native}
\justifying


Native measurements cover the Intel x86 platform (Tests T5--T8) and compare baseline (stock/control) boots against the optimized configurations applied in this part of the project. Since the optimization phase primarily targeted early boot, interpretation focuses on the \textit{loader} and \textit{kernel} timings; total boot time is still reported for completeness but is often dominated by firmware behavior and userspace service start-up (especially on USB boots). The optimizations are focused on kernel and loader time.

\begin{table}[H]
\centering
\caption{Native results for targeted phases (mean $\pm$ std. dev., $n=5$).}
\label{tab:native_target_results}
\begin{tabular}{lllllll}
\hline
\textbf{Test} & \textbf{OS} & \textbf{Storage} & \textbf{Config} & \textbf{Loader (s)} & \textbf{Kernel (s)} & \textbf{Total (s)} \\
\hline
T6 & Arch & NVMe & Stock & 1.170 $\pm$ 0.014 & 0.782 $\pm$ 0.010 & 22.602 $\pm$ 2.430 \\
T6 & Arch & NVMe & Optimized & 0.375 $\pm$ 0.005 & 0.764 $\pm$ 0.013 & 21.215 $\pm$ 0.392 \\
\hline
T5 & Arch & USB & Stock & 2.601 $\pm$ 0.144 & 0.780 $\pm$ 0.006 & 21.043 $\pm$ 1.990 \\
T5 & Arch & USB & Optimized & 1.250 $\pm$ 0.436 & 0.756 $\pm$ 0.004 & 17.375 $\pm$ 1.450 \\
\hline
T8 & Debian & NVMe & Stock & 4.437 $\pm$ 0.319 & 3.533 $\pm$ 0.041 & 22.124 $\pm$ 0.430 \\
T8 & Debian & NVMe & Optimized & 0.499 $\pm$ 0.010 & 2.506 $\pm$ 0.033 & 34.836 $\pm$ 0.056 \\
\hline
T7 & Debian & USB & Stock (recreated) & 2.776 $\pm$ 0.021 & 44.719 $\pm$ 0.346 & 155.868 $\pm$ 0.51 \\
T7 & Debian & USB & Optimized & 1.862 $\pm$ 0.367 & 6.570 $\pm$ 0.842 & 134.117 $\pm$ 13.6 \\
\hline
\end{tabular}
\end{table}

\subsubsection{Intel Arch Linux (Native)}
On NVMe (T6), the optimized configuration produced a clear improvement in loader time, dropping from $1.170 \pm 0.014$~s to $0.375 \pm 0.005$~s (about 68\% reduction). Kernel time changed only slightly, from $0.782 \pm 0.010$~s to $0.764 \pm 0.013$~s (about 2\% reduction), which is consistent with an already fast baseline kernel path on this distribution. The total time decreased from $22.602 \pm 2.430$~s to $21.215 \pm 0.392$~s; the larger baseline variance indicates that non-target phases (primarily firmware behavior) still drive a meaningful portion of total wall-time variability.\\

On USB (T5), improvements were more visible at the system level because the baseline boot is more sensitive to storage throughput and early-stage I/O. Loader time decreased from $2.601 \pm 0.144$~s to $1.250 \pm 0.436$~s, and total boot time decreased from $21.043 \pm 1.990$~s to $17.375 \pm 1.450$~s. Kernel time remained sub-second in both cases, indicating that for this Arch setup the dominant constraints on USB are not kernel execution time, but the surrounding I/O-heavy phases (initrd/userspace) that scale with flash drive read performance.

\subsubsection{Intel Debian 12 (Native)}
On NVMe (T8), the optimization significantly reduced the loader and kernel phases: loader time decreased from $4.437 \pm 0.319$~s to $0.499 \pm 0.010$~s (about 89\% reduction), and kernel time decreased from $3.533 \pm 0.041$~s to $2.506 \pm 0.033$~s (about 29\% reduction). However, the total time increased from $22.124 \pm 0.430$~s to $34.836 \pm 0.056$~s because the non-target phases increased substantially in the measured runs. This outcome reinforces a key interpretation rule for the native NVMe case: early-boot improvements (loader/kernel) can be masked if firmware time and userspace service start-up are not held constant across installations and service sets. For this report, the loader/kernel improvements are the relevant indicators of optimization effectiveness, while the total-time regression is treated as evidence that userspace/firmware variability can become the dominant factor in that configuration when others are reduced.\\

On USB (T7), the recreated baseline exhibited a very large kernel phase ($44.719 \pm 0.346$~s), consistent with an I/O-limited early boot path where module loading, probing, and initramfs activity are heavily constrained by flash read speed. After optimization, kernel time dropped to $6.570 \pm 0.842$~s (about 85\% reduction), and loader time also decreased from $2.776 \pm 0.021$~s to $1.862 \pm 0.367$~s corresponding to an approximote reduction of $0.914$~s (about $32.9\%$). Total boot time improved from $155.868 \pm 0.503$~s to $134.117 \pm 13.672$~s. The larger post-optimization variance is explained by the userspace phase remaining dominant on USB and fluctuating across trials, which is expected when the storage device remains the primary bottleneck even after early boot trimming.



\subsection{Virtual Environment (ARM -- Parallels Desktop)}
\justifying

The virtualized ARM environment under Parallels Desktop on Apple Silicon yielded substantial performance improvements through targeted kernel optimization and hypervisor-specific tuning. Results are presented for both Arch Linux ARM and Debian 12 ARM64 guests across NVMe and USB storage configurations. All measurements represent mean values over five cold-boot trials, with optimization focused on kernel and loader phases as the primary controllable components within the VM environment.

\subsubsection{Arch Linux ARM (aarch64)}

Arch Linux ARM demonstrated the most significant optimization gains among all tested configurations, with kernel time reductions exceeding 60\% across both storage media.

\paragraph{NVMe Configuration}
\justifying

The NVMe configuration achieved a total boot time reduction from 6.19~s to 3.65~s, representing a \textbf{41\% improvement} in overall boot performance. Phase-level analysis reveals the following contributions:

\begin{table}[H]
\centering
\caption{Arch Linux ARM boot time breakdown (NVMe)}
\label{tab:arch_arm_nvme_results}
\begin{tabular}{lcccc}
\hline
\textbf{Phase} & \textbf{Stock (s)} & \textbf{Optimized (s)} & \textbf{$\Delta$ (s)} & \textbf{Improvement} \\
\hline
Firmware & 1.12 $\pm$ 0.16 & 0.66 $\pm$ 0.02 & $-0.46$ & 41\% \\
Loader & 0.21 $\pm$ 0.01 & 0.19 $\pm$ 0.00 & $-0.02$ & 10\% \\
Kernel & 2.08 $\pm$ 0.01 & 0.72 $\pm$ 0.02 & $-1.36$ & \textbf{65\%} \\
Initrd & 1.33 $\pm$ 0.06 & 0 & $-1.33$ & Eliminated \\
Userspace & 1.45 $\pm$ 0.01 & 2.08 $\pm$ 0.03 & $+0.63$ & $-43\%$ \\
\hline
\textbf{Total} & \textbf{6.19 $\pm$ 0.15} & \textbf{3.65 $\pm$ 0.06} & \textbf{$-2.54$} & \textbf{41\%} \\
\hline
\end{tabular}
\end{table}

The kernel phase contributed the largest absolute reduction (1.36~s, representing 54\% of total improvement). Combined with the elimination of the initrd phase (1.33~s), kernel-level optimizations accounted for 2.69~s of the 2.54~s total improvement. The apparent negative efficiency stems from increased userspace time (1.45~s $\rightarrow$ 2.08~s), likely due to built-in drivers and services that previously initialized during the initrd phase now executing in userspace.\\

The primary optimization—\texttt{libahci.ignore\_sss=1}—addressed the Parallels-specific SATA port probing bottleneck and contributed approximately 1.2--1.4~s of the kernel time reduction. Secondary custom kernel trimming provided an additional 100--150~ms improvement.

\paragraph{USB Flash Drive Configuration}
\justifying

The USB configuration achieved similar relative improvements with a total boot time reduction from 7.44~s to 4.48~s (\textbf{40\% faster}):

\begin{table}[H]
\centering
\caption{Arch Linux ARM boot time breakdown (USB)}
\label{tab:arch_arm_usb_results}
\begin{tabular}{lcccc}
\hline
\textbf{Phase} & \textbf{Stock (s)} & \textbf{Optimized (s)} & \textbf{$\Delta$ (s)} & \textbf{Improvement} \\
\hline
Firmware & 0.72 $\pm$ 0.06 & 0.68 $\pm$ 0.02 & $-0.04$ & 6\% \\
Loader & 1.57 $\pm$ 0.45 & 0.57 $\pm$ 0.01 & $-1.00$ & \textbf{64\%} \\
Kernel & 2.06 $\pm$ 0.01 & 0.82 $\pm$ 0.02 & $-1.24$ & \textbf{60\%} \\
Initrd & 1.39 $\pm$ 0.09 & 0 & $-1.39$ & Eliminated \\
Userspace & 1.70 $\pm$ 0.15 & 2.40 $\pm$ 0.10 & $+0.70$ & $-41\%$ \\
\hline
\textbf{Total} & \textbf{7.44 $\pm$ 0.59} & \textbf{4.48 $\pm$ 0.10} & \textbf{$-2.96$} & \textbf{40\%} \\
\hline
\end{tabular}
\end{table}

The USB configuration exhibited a particularly large loader time reduction (1.00~s, 64\%), likely attributable to reduced loader complexity and faster device initialization when combined with the optimized kernel. Kernel time improvement (60\%) remained consistent with the NVMe configuration, confirming that the AHCI/SSS optimization is storage-independent within the virtual environment.

\subsubsection{Debian 12 ARM64 (arm64)}

Debian 12 ARM64 measurements validated the transferability of the optimization strategy across distributions, achieving kernel time reductions of 61--66\% despite using an older kernel baseline (6.1.x versus Arch's 6.18.x).

\paragraph{NVMe Configuration}
\justifying

Debian NVMe achieved a total boot time reduction from 7.33~s to 5.43~s (\textbf{26\% improvement}):

\begin{table}[H]
\centering
\caption{Debian 12 ARM64 boot time breakdown (NVMe)}
\label{tab:debian_arm_nvme_results}
\begin{tabular}{lcccc}
\hline
\textbf{Phase} & \textbf{Stock (s)} & \textbf{Optimized (s)} & \textbf{$\Delta$ (s)} & \textbf{Improvement} \\
\hline
Kernel & 2.94 $\pm$ 0.02 & 1.01 $\pm$ 0.23 & $-1.93$ & \textbf{66\%} \\
Userspace & 4.39 $\pm$ 0.20 & 4.42 $\pm$ 0.31 & $+0.03$ & $-1\%$ \\
\hline
\textbf{Total} & \textbf{7.33 $\pm$ 0.21} & \textbf{5.43 $\pm$ 0.26} & \textbf{$-1.90$} & \textbf{26\%} \\
\hline
\end{tabular}
\end{table}

Debian's kernel time reduction (1.93~s, 66\%) represents the highest relative improvement observed across all ARM configurations. The absolute kernel time in the optimized state (1.01~s) is approximately 280~ms higher than Arch's optimized kernel time (0.72~s), consistent with expected differences between Debian's stable kernel (6.1.158) and Arch's recent kernel (6.18.2). Userspace time remained effectively unchanged, indicating that the kernel optimizations did not introduce userspace-level regressions.

\paragraph{USB Flash Drive Configuration}
\justifying

Debian USB achieved a total boot time reduction from 9.92~s to 8.40~s (\textbf{15\% improvement}):

\begin{table}[H]
\centering
\caption{Debian 12 ARM64 boot time breakdown (USB)}
\label{tab:debian_arm_usb_results}
\begin{tabular}{lcccc}
\hline
\textbf{Phase} & \textbf{Stock (s)} & \textbf{Optimized (s)} & \textbf{$\Delta$ (s)} & \textbf{Improvement} \\
\hline
Kernel & 2.54 $\pm$ 0.03 & 1.00 $\pm$ 0.26 & $-1.54$ & \textbf{61\%} \\
Userspace & 7.38 $\pm$ 0.50 & 7.40 $\pm$ 0.29 & $+0.02$ & 0\% \\
\hline
\textbf{Total} & \textbf{9.92 $\pm$ 0.49} & \textbf{8.40 $\pm$ 0.22} & \textbf{$-1.52$} & \textbf{15\%} \\
\hline
\end{tabular}
\end{table}

The USB configuration shows a 61\% kernel time reduction while maintaining stable userspace performance. The lower total improvement percentage (15\% versus 26\% on NVMe) reflects the larger absolute userspace time in the USB baseline (7.38~s), which dilutes the relative impact of kernel-level gains.

\subsubsection{Cross-Distribution Analysis}

Comparing Arch Linux ARM and Debian 12 ARM64 under identical virtual hardware reveals distribution-level differences in both baseline performance and optimization response:

\begin{table}[H]
\centering
\caption{Distribution comparison: kernel time (optimized, NVMe)}
\label{tab:distro_comparison}
\begin{tabular}{lcccc}
\hline
\textbf{Distribution} & \textbf{Kernel Ver.} & \textbf{Stock (s)} & \textbf{Optimized (s)} & \textbf{$\Delta K$ (s)} \\
\hline
Arch Linux ARM & 6.18.2 & 2.08 & 0.72 & $-1.36$ (65\%) \\
Debian 12 ARM64 & 6.1.158 & 2.94 & 1.01 & $-1.93$ (66\%) \\
\hline
\textbf{Difference} & --- & $+0.86$ & $+0.29$ & $-0.57$ \\
\hline
\end{tabular}
\end{table}

Debian's stock kernel time is 41\% higher than Arch's stock kernel time (2.94~s versus 2.08~s), but after optimization, Debian's kernel time is only 40\% higher (1.01~s versus 0.72~s). Both distributions converge to sub-second kernel times after applying the \texttt{libahci.ignore\_sss=1} mitigation and custom kernel trimming, confirming that the Parallels AHCI/SSS bottleneck affects both distributions equally and responds similarly to the same optimization strategy.

\subsubsection{Optimization Attribution}

The measured improvements can be decomposed into contributions from specific optimization categories:

\begin{itemize}
    \item \textbf{AHCI/SSS mitigation (\texttt{libahci.ignore\_sss=1}):} Primary contributor, accounting for approximately 1.2--1.5~s of kernel time reduction across all configurations. This addresses the virtual AHCI controller's staggered spin-up behavior and eliminates serialized probing of unused SATA ports.
    
    \item \textbf{Custom kernel trimming:} Secondary contributor, providing 100--200~ms of additional kernel time reduction through removal of unused drivers, subsystems, and initialization overhead.
    
    \item \textbf{Initrd elimination (Arch only):} Complete removal of the initrd phase (1.3--1.4~s) through built-in driver integration. This optimization is distribution-specific; Debian's boot path did not report a distinct initrd phase in the baseline measurements.
    
    \item \textbf{Loader optimization (Arch only):} Modest improvements (10--64\%) achieved through systemd-boot adoption and boot path simplification. Debian measurements did not report loader times.
\end{itemize}

\subsubsection{Statistical Confidence and Measurement Variability}

Standard deviations across five-trial measurements indicate high reproducibility for most boot phases. Kernel time measurements exhibited particularly low variance (standard deviations $\leq$0.03~s in most cases), validating the consistency of the optimization effects. Userspace time showed higher variability (standard deviations up to 0.50~s), reflecting the non-deterministic nature of service initialization and network-dependent startup tasks.

The largest standard deviation was observed in Arch ARM USB loader time (stock: 1.57~s $\pm$ 0.45~s), likely due to variable USB device enumeration and initialization timing in the virtual environment. After optimization, loader variability decreased significantly (0.57~s $\pm$ 0.01~s), suggesting that the simplified boot path also improved timing consistency.






