
\begin{abstract}
This report extends the baseline boot time study from Part~1 by changing and evaluating optimizations. Experiments cover eight different Linux configurations including native x86\_64 systems and an ARM64 Parallels virtual machine, for distros: Debian~12 and Arch Linux, and for storage USB and NVMe. The optimization range is intentionally limited to make it measurable and controllable across the full test matrix: the bootloader (\textbf{Loader}) and the Linux kernel (\textbf{Kernel}) as reported by \texttt{systemd-analyze}. Each change is evaluated using five cold-boot trials before and after modification, with results summarized using mean and sample standard deviation. Applied optimizations include simplifying the boot path by transitioning from GRUB to \texttt{systemd-boot}, reducing kernel overhead through configuration trimming and driver integration decisions (with USB boot-safety constraints), adjusting compression choices for the kernel/initramfs, and mitigating a platform-specific Parallels AHCI bottleneck. Firmware and userspace times are recorded for context and traceability, but they are not treated as primary optimization targets.
\end{abstract}
